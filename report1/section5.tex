⑤つぎのシステムについて,問いに答えよ.
$$
\frac{\mathrm{d}}{\mathrm{d} t} \boldsymbol{x}(t)=\boldsymbol{A} \boldsymbol{x}(t)+\boldsymbol{b} u(t), \quad \boldsymbol{A}=\left[\begin{array}{cc}
0 & 1 \\
-3 & -4
\end{array}\right], \quad \boldsymbol{b}=\left[\begin{array}{l}
1 \\
1
\end{array}\right]
$$

a)\,逆ラプラス変換を利用して$e^{At}$を求めよ.


b)\,$\boldsymbol{x}(0)=\left[\begin{array}{ll}-1 & 1\end{array}\right]^T, u(t) \equiv 1$
としたときのシステムの解軌道$x(t)$を求めよ.


$$
\begin{aligned}
    a)\,
    \mathrm{e}^{A t} & =\mathcal{L}^{-1}\left\{(s I-A)^{-1}\right\}=\mathcal{L}^{-1}\left\{\left[\begin{array}{cc}
    s & -1 \\
    3 & s+4
    \end{array}\right]^{-1}\right\}=\mathcal{L}^{-1}\left\{\frac{1}{s^2+4 s+3}\left[\begin{array}{cc}
    s+4 & 1 \\
    -3 & s
    \end{array}\right]\right\} \\
    & =\mathcal{L}^{-1}\left\{\left[\begin{array}{cc}
    \frac{3}{2} \frac{1}{s+1}-\frac{1}{2} \frac{1}{s+3} & \frac{1}{2} \frac{1}{s+1}-\frac{1}{2} \frac{1}{s+3} \\
    -\frac{3}{2} \frac{1}{s+1}+\frac{3}{2} \frac{1}{s+3} & -\frac{1}{2} \frac{1}{s+1}+\frac{3}{2} \frac{1}{s+3}
    \end{array}\right]\right\}=\left[\begin{array}{cc}
    \frac{3}{2} \mathrm{e}^{-t}-\frac{1}{2} \mathrm{e}^{-3 t} & \frac{1}{2} \mathrm{e}^{-t}-\frac{1}{2} \mathrm{e}^{-3 t} \\
    -\frac{3}{2} \mathrm{e}^{-t}+\frac{3}{2} \mathrm{e}^{-3 t} & -\frac{1}{2} \mathrm{e}^{-t}+\frac{3}{2} \mathrm{e}^{-3 t}
    \end{array}\right]
\end{aligned}
$$


$$
\begin{aligned}
    b)\,
    & \int_0^t \mathrm{e}^{A(t-\tau)} b u(\tau) d \tau=\int_0^t\left[\begin{array}{ll}
    \frac{3}{2} \mathrm{e}^{\tau-t}-\frac{1}{2} \mathrm{e}^{3(\tau-t)} & \frac{1}{2} \mathrm{e}^{\tau-t}-\frac{1}{2} \mathrm{e}^{3(\tau-t)} \\
    -\frac{3}{2} \mathrm{e}^{\tau-t}+\frac{3}{2} \mathrm{e}^{3(\tau-t)} & -\frac{1}{2} \mathrm{e}^{\tau-t}+\frac{3}{2} \mathrm{e}^{3(\tau-t)}
    \end{array}\right]\left[\begin{array}{l}
    1 \\
    1
    \end{array}\right] d \tau \\
    & =\int_0^t\left[\begin{array}{c}
    2 \mathrm{e}^{-t} \mathrm{e}^\tau-\mathrm{e}^{-3 t} \mathrm{e}^{3 \tau} \\
    -2 \mathrm{e}^{-t} \mathrm{e}^\tau+3 \mathrm{e}^{-3 t} \mathrm{e}^{3 \tau}
    \end{array}\right] d \tau=\left[\left[\begin{array}{c}
    2 \mathrm{e}^{-t} \mathrm{e}^\tau-\mathrm{e}^{-3 t} \frac{1}{3} \mathrm{e}^{3 \tau} \\
    -2 \mathrm{e}^{-t} \mathrm{e}^\tau+\mathrm{e}^{-3 t} \mathrm{e}^{3 \tau}
    \end{array}\right]\right]_0^t \\
    & =\left[\begin{array}{c}
    2-\frac{1}{3}-\left(2 \mathrm{e}^{-t}-\frac{1}{3} \mathrm{e}^{-3 t}\right) \\
    -2+1-\left(-2 \mathrm{e}^{-t}+\mathrm{e}^{-3 t}\right)
    \end{array}\right]=\left[\begin{array}{c}
    \frac{5}{3}-2 \mathrm{e}^{-t}+\frac{1}{3} \mathrm{e}^{-3 t} \\
    -1+2 \mathrm{e}^{-t}-\mathrm{e}^{-3 t}
    \end{array}\right]
    \end{aligned}
$$
$$
\begin{aligned}
x(t) & =\mathrm{e}^{A t} x(0)+\int_0^t \mathrm{e}^{A(t-\tau)} b u(\tau) d \tau=\left[\begin{array}{cc}
\frac{3}{2} \mathrm{e}^{-t}-\frac{1}{2} \mathrm{e}^{-3 t} & \frac{1}{2} \mathrm{e}^{-t}-\frac{1}{2} \mathrm{e}^{-3 t} \\
-\frac{3}{2} \mathrm{e}^{-t}+\frac{3}{2} \mathrm{e}^{-3 t} & -\frac{1}{2} \mathrm{e}^{-t}+\frac{3}{2} \mathrm{e}^{-3 t}
\end{array}\right]\left[\begin{array}{l}
1 \\
1
\end{array}\right]+\left[\begin{array}{c}
\frac{5}{3}-2 \mathrm{e}^{-t}+\frac{1}{3} \mathrm{e}^{-3 t} \\
-1+2 \mathrm{e}^{-t}-\mathrm{e}^{-3 t}
\end{array}\right] \\
& =\left[\begin{array}{c}
2 \mathrm{e}^{-t}-\mathrm{e}^{-3 t} \\
-2 \mathrm{e}^{-t}+3 \mathrm{e}^{-3 t}
\end{array}\right]+\left[\begin{array}{l}
\frac{5}{3}-2 \mathrm{e}^{-t}+\frac{1}{3} \mathrm{e}^{-3 t} \\
-1+2 \mathrm{e}^{-t}-\mathrm{e}^{-3 t}
\end{array}\right]=\left[\begin{array}{c}
\frac{5}{3}-\frac{2}{3} \mathrm{e}^{-3 t} \\
-1+2 \mathrm{e}^{-3 t}
\end{array}\right]
\end{aligned}
$$

\newpage
